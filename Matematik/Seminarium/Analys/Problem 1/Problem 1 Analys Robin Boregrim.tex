\documentclass[11pt]{article}
\usepackage[utf8]{inputenc}
\usepackage{centernot}
\usepackage[parfill]{parskip}
\usepackage{amsmath}
\begin{document}
\title{Analys Problem 1}
\author{Robin Boregrim}
\maketitle
\renewcommand{\contentsname}{Innehållsförteckning}
\tableofcontents
\newpage
\section{Uppgiften}
Beräkna gränsvärdet av 
$$x_n = \frac{\sqrt{1+2n}-3}{\sqrt{n}-2}$$
när $n$ går mot oändligheten.
\section{Lösning}
\subsection{Observationer}
Om vi observerar gränsvärdet för $x_n$
\begin{equation} \label{eq:start}
\lim_{n\rightarrow \infty} \Bigg(\frac{\sqrt{1+2n}-3}{\sqrt{n}-2} \Bigg)
\end{equation}
ser vi att den är ett gränsvärde av typen "$\frac{\infty}{\infty}$" eftersom
$$\sqrt{1+2n}-3 \rightarrow \infty$$
och 
$$\sqrt{n}-2 \rightarrow \infty.$$
Detta betyder att vi inte bara kan räkna ut gränsvärdet direkt eftersom "$\frac{\infty}{\infty}$" kan bli vad som helst. Vi måste därför skriva om (\ref{eq:start}) så man kan räkna ut gränsvärdet.
\subsection{Uträkningar}
Vi börjar med att dela täljaren i (\ref{eq:start}) men nämnaren så långt som det går, då får vi:
$$\lim_{n\rightarrow \infty} \Bigg( \sqrt{\frac{1}{n} + 2} + \frac{2\cdot\sqrt{\frac{1}{n} + 2} -3}{\sqrt{n}-2}\Bigg).$$

Vi vet att när $n \rightarrow \infty$ så
\begin{equation} \label{eq:zero}
\frac{1}{n} \rightarrow 0
\end{equation}
och
\begin{equation} \label{eq:inf}
\sqrt{n} \rightarrow \infty.
\end{equation}
\newpage
Av (\ref{eq:zero}) följer att när $n \rightarrow \infty$
$$\sqrt{\frac{1}{n} + 2} \rightarrow \sqrt{0 + 2} = \sqrt{2}$$
och av (\ref{eq:zero}) och (\ref{eq:inf}) följer att när $n \rightarrow \infty$
$$\frac{2\cdot\sqrt{\frac{1}{n} + 2} -3}{\sqrt{n}-2} \rightarrow \frac{2\sqrt{2} -3}{\infty} = 0.$$
Detta betyder att 
$$\lim_{n\rightarrow \infty} \Bigg( \sqrt{\frac{1}{n} + 2} + \frac{2\cdot\sqrt{\frac{1}{n} + 2} -3}{\sqrt{n}-2}\Bigg) = \sqrt{2} + 0 = \sqrt{2}.$$
\subsection{Svar}
Gränsvärdet av 
$$x_n = \frac{\sqrt{1+2n}-3}{\sqrt{n}-2}$$
när $n \rightarrow \infty$ är $$\sqrt{2}.$$
\end{document}