\documentclass[11pt]{article}
\usepackage[utf8]{inputenc}
\usepackage{centernot}
\usepackage[parfill]{parskip}
\usepackage{amsmath}
\usepackage{amssymb}
\begin{document}
\title{Analys Problem 2}
\author{Robin Boregrim}
\maketitle
\renewcommand{\contentsname}{Innehållsförteckning}
\tableofcontents
\newpage
\section{Uppgiften}
För vilka värden på $a> 0$ har ekvationen $a^x = x$ lösningar?
\section{Lösning}
\subsection{Omskrivning}
För att kunna lösa ut $a$ så skriver vi om ekvationen $a^x = x$ så att $a$ blir en funktion av $x$.
\begin{equation}\label{start}
a^x = x \Rightarrow
\end{equation}
Om vi antar att $x \centernot = 0$
$$ (a^x)^{\frac{1}{x}} = x^{\frac{1}{x}}, x \centernot= 0 \Leftrightarrow$$ 
\begin{equation}\label{re}
 a = x^{\frac{1}{x}}, x \centernot= 0
\end{equation}
Ekvation (\ref{re}) inte är definerad för $x=0$, detta skulle vara ett problem om (\ref{start}) hade lösningar för $x=0$ men eftersom 
$$a^0 = 1, \forall a\in {\rm I\!R}$$ så $a^0$ kommer aldrig vara lika med $0$. $x = 0$ är därför aldrig en lösning på (\ref{start}).\\
En annan sak värd att notera är att eftersom $a^x$ kommer vara positivt för alla $x$ kommer ekvationen $a^x = x$ inte ha några lösningar för negativa $x$. Vänsterledet $a^x$ skulle vara positivt och högerledet $x$ skulle vara negativt för alla negativa $x$ dvs 
$$a^x > 0, \forall x \in {\rm I\!R} < 0$$
$$x < 0, \forall x \in {\rm I\!R} < 0$$
$$a^x \centernot = x, \forall x \in {\rm I\!R} < 0.$$
Vi vet därför att definitionsmängden på x är
$$D_f = \{ x > 0 \} .$$
\subsection{Derivataräkning}
Nu vet vi att $a$ beskrivs av funktionen $a = x^{\frac{1}{x}}$ så vi räknar ut derviatan av den funktionen för att sen ta reda på eventuella extrempunkter.
$$a' = \frac{d}{dx}(x^{\frac{1}{x}})$$
$$= \frac{d}{dx}(e^{\ln(x^{\frac{1}{x}})})$$
$$= \frac{d}{dx}(e^{\frac{1}{x}\ln(x)})$$
$$=\frac{d}{dx}(\frac{\ln x}{x}) \cdot e^{\frac{1}{x}\ln x}$$
$$=(\frac{\ln x}{x})' \cdot x^{\frac{1}{x}}$$
$$=\frac{x \cdot \frac{d}{dx}(\ln x) - \ln x \cdot \frac{d}{dx} (x)}{x^2} \cdot x^{\frac{1}{x}}$$
$$=\frac{x\cdot \frac{1}{x} - \ln x \cdot 1}{x^2} \cdot x^{\frac{1}{x}}$$

\begin{equation}\label{prim}
a' =\frac{1 - \ln x}{x^2} \cdot x^{\frac{1}{x}}
\end{equation}
Nu behöver vi räkna ut för vilka $x$ som $a'=0$ för att hitta eventuella extrempunkter.
Varken $x^{\frac{1}{x}}$ eller $\frac{1}{x^2}$ kan bli 0, vilket betyder att om (\ref{prim})$=0$ måste 
$$1-\ln x = 0$$
$$1 = \ln x$$
$$e^1 = e^{\ln x}$$
$$e^1 = e^{\ln x}$$
$$x = e$$
Nu när vi vet att $a'$ endast har en rot som är $e$ vill vi veta om roten är en maximi-, mini- eller terrasspunkt.\\
Både $x^{\frac{1}{x}}$ och $\frac{1}{x^2}$ är alltid possitiva för $\forall x\in {\rm I\!R} \cup D_f$.
Detta betyder att $a'$ är positivt eller negativt beroende på om $1-\ln x$ är positivt eller negativt.\\
Eftersom funktionen $\ln x$ är strängt växande och $\ln e = 1$ är $\ln x < 1$ om $x < e$ och $\ln x > 1$ om $x > e$. \\
Vilket i sin tur betyder att $a'$ är positiv för $x < e$ och negativ för $x > e$. Roten $x = e$ är därför en global maximipunkt för (\ref{re}).\\
Av detta följer då att $a \leq e^{\frac{1}{e}}.$ Eftersom $a$ var större än noll per definition får vi:
$$0 < a \leq e^{\frac{1}{e}}.$$
\subsection{Svar}
Ekvationen $a^x = x$ har lösningar i internvallet $$0 < a \leq e^{\frac{1}{e}}.$$
\end{document}