\documentclass[11pt]{article}
\usepackage[utf8]{inputenc}
\usepackage{centernot}
\usepackage[parfill]{parskip}
\usepackage{amsmath}
\begin{document}
\title{Algebra Problem 3}
\author{Robin Boregrim}

\maketitle
\renewcommand{\contentsname}{Innehållsförteckning}
\tableofcontents
\newpage
\section{Uppgiften}
Undersök vilka $x$ som uppfyller olikheten $$\frac{x^2-1}{x-4} \geq x.$$
\section{Lösning}
\subsection{Beräkningar}
Det första vi kan observera är att $\frac{x^2-1}{x-4}$ inte är definerat för $x = 4$ efter som $\frac{x^2-1}{x-4}$ bildar en division med noll då $x = 4$, $x = 4$ är därför inte en lösning.\\
Därefter kan vi multiplicera båda led med $x-4$ och dela in ekvationen i två fall beroende på om $x-4$ är positivt eller negativt, dvs om $x$ är större eller mindre än $4$. Vi förenklar även fallen.\\
Fall A ($x > 4$):
$$(x-4)\frac{x^2-1}{x-4} \geq (x-4)x$$
$$x^2-1 \geq x^2 - 4x$$
[subtrahera $x^2$ från båda led]
$$-1 \geq - 4x$$
[multiplicera båda led med $-\frac{1}{4}$]
$$\frac{1}{4} \leq x.$$
Fall B ($x < 4$):
$$(x-4)\frac{x^2-1}{x-4} \leq (x-4)x$$
$$x^2-1 \leq x^2 - 4x$$
[subtrahera $x^2$ från båda led]
$$-1 \leq - 4x$$
[multiplicera båda led med $-\frac{1}{4}$]
$$\frac{1}{4} \geq x.$$
\newpage
\subsection{Analys}
Fall A säger; då om $x$ är större än 4 måste $x$ även vara större eller lika med $\frac{1}{4}$, vilket är sant för alla $x > 4$.\\
Fall B säger; då om $x$ är mindre än 4 måste $x$ även vara mindre eller lika med $\frac{1}{4}$, vilket är sant för alla $x \leq \frac{1}{4}$.\\
Fall A och B kombinerade ger då:
$$x > 4 \vee x \leq \frac{1}{4}.$$
\subsection{Svar}
Olikheten $$\frac{x^2-1}{x-4} \geq x$$
uppfylls av alla $x$ som är större en fyra eller mindre eller lika med $\frac{1}{4}$, dvs
$$x > 4 \vee x \leq \frac{1}{4}.$$


\end{document}