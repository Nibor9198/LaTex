\documentclass[11pt]{article}
\usepackage[utf8]{inputenc}
\usepackage{centernot}
\usepackage[parfill]{parskip}
\usepackage{amsmath}
\usepackage{amssymb}
\begin{document}
\title{Algebra Problem 4}
\author{Robin Boregrim}
\maketitle
\renewcommand{\contentsname}{Innehållsförteckning}
\tableofcontents
\newpage
\section{Uppgiften}
Lös för varje reellt tal $a$ ekvationssystemet\\
$$
\left\{\begin{array}{ccccc}
 x & -y & +z & = & a \\
 x & +y & +3z & = & a+2\\
 2x & -2y & (a+1)z & = & a+1
\end{array}\right.
$$



\section{Lösning}
\subsection{Matriser och Radoperationer}
Först så gör vi om ekvationssystemet till en matris och använder tillåtna radoperationer för att sedan försöka lösa ut ekvationssystemet med Gauss-Jordan eliminering.
$$
\left(\begin{array}{ccc|c}  
 1 & -1 & 1 & a \\
 1 & 1 & 3 & a+2\\
 2 & -2 & a+1 & a+1
\end{array}\right)
$$
Vi börjar med att ta bort första raden ifrån andra raden
$$
\left(\begin{array}{ccc|c}  
 1 & -1 & 1 & a \\
 0 & 2 & 2 & 2\\
 2 & -2 & a+1 & a+1
\end{array}\right)
$$
Sen tar vi bort två gånger första raden ifrån tredje raden
$$
\left(\begin{array}{ccc|c}  
 1 & -1 & 1 & a \\
 0 & 2 & 2 & 2\\
 0 & 0 & a-1 & 1-a
\end{array}\right)
$$

\subsection{Analys av ekvationssystemet}
Om vi gör om den nu förenklade matrisen tillbaka till ett ekvationssystem får vi:
$$
\left\{\begin{array}{ccccc}
 x & -y & +z & = & a \\
  & 2y & +2z & = & 2\\
  &  & (a-1)z & = & 1-a
\end{array}\right.
\begin{array}{c}
(1)\\ (2) \\ (3)
\end{array}
$$\\\\
Av (3) kan vi observera att vi får 2 fall.\\\\
Fall 1: \\
	Om $a \centernot = 1$ får vi ett ekvationssystem med oändligt många lösningar som beror på $a$ där (3) ger $$(a-1)z = (1-a).$$
Fall 2:\\
Om $a = 1$ får vi ett ekvationssystem med oändligt många lösningar som beror på $z$ då (3) ger:
	$$(a-1)z = (1-a), a = 1 \Leftrightarrow $$
	$$ (1-1)z = (1-1) \Leftrightarrow $$
	$$ 0=0.$$
\subsection{Lösning av ekvationssystemet}
\subsubsection{Fall 1}
När $a \centernot = 1$ ger (3):

$$z = \frac{1-a}{a-1} = \frac{1-a}{-1 \cdot (1-a)} = \frac{1}{-1} = -1$$
$$z = -1$$
Av detta följer att (2) ger:
$$2y + 2 \cdot(-1) = 2$$
$$y-1=1$$
$$y=2$$
Av detta följer att (1) ger:
$$x -2 + (-1) = a$$
$$x-3 = a$$
$$x = a+3$$\\
Då får vi ekvationssystemet
$$
\left\{\begin{array}{ccc}
 x & = & a+3\\
 y & = & 2\\
 z & = & -1
\end{array}\right.
$$
\subsubsection{Fall 2}
Om vi sätter in $a = 1$ i ekvationssystemet får vi
$$
\left\{\begin{array}{ccccc}
 x & -y & +z & = & 1 \\
  & 2y & +2z & = & 2\\
  &  & (1-1)z & = & 1-1
\end{array}\right.
,$$
vilket kan förenklas till
$$
\left\{\begin{array}{ccccc}
 x & -y & +z & = & 1 \\
  & 2y & +2z & = & 2\\
  &  & 0 & = & 0
\end{array}\right.
\begin{array}{c}
(1)\\ (2) \\ (3)
\end{array}
 .$$
 
 Ekvationen (3) blir då $0 = 0$ vilket alltid stämmer oberoende av några variabler. Detta betyder att vi har 2 ekvationer som beror på 3 variabler vilket i sin tur betyder att vi kommer ha oändligt många lösningar som beror på en variabel som jag kallar för $b$.\\
 Vi låter $z = b$.\\
 Av detta följer att (2) ger
 $$2y + 2b= 2$$
 $$y+b=1$$
 $$y= 1-b.$$
 Och av detta följer att (1) ger
 $$x -(1-b)+b = 1$$
 $$x - 1 +2b = 1$$
 $$x = 2 - 2b.$$\\
 Och vi kan sammanställa detta i ett ekvationssystem
 $$
\left\{\begin{array}{ccc}
 x & = & 2-2b	\\
 y & = & 1-b	\\
 z & = & b
\end{array}\right.
$$
\subsection{Svar}
Om $a \centernot = 1$ är lösningarna till är lösningarna till ekvationssysytemet
$$
\left\{\begin{array}{ccc}
 x & = & a+3\\
 y & = & 2\\
 z & = & -1
\end{array}\right.
.$$
Om $a=1$ så är lösningarna på ekvationssystemet
$$
\left\{\begin{array}{ccc}
 x & = & 2-2b	\\
 y & = & 1-b	\\
 z & = & b
\end{array}\right.
$$
för alla godtyckliga $b$.
\end{document}