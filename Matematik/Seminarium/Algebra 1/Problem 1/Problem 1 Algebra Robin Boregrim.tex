\documentclass[11pt]{article}
\usepackage[utf8]{inputenc}
\begin{document}
\title{Algebra Problem 1}
\author{Robin Boregrim}
\maketitle
\tableofcontents
\newpage

\section{Uppgiften}
Lös rotekvationen $$2\sqrt{x+5}-\sqrt{x-3} = 5$$ för alla reella x.
\section{Lösning}
\subsection{Förenkling}
För att lösa rotekvationen
$$2\sqrt{x+5}-\sqrt{x-3} = 5$$ 
vill vi få bort rottecken, därför börjar vi med att kvadrera båda led
$$(2\sqrt{x+5}-\sqrt{x-3})^2 = 5^2$$ 
$$4(x+5) - 4\sqrt{x+5}\sqrt{x-3} +(x-3) = 25$$
forstätter med att expandera $4(x+5)$ och $(x-3)$
$$4x+ 20 + x - 3 - 4\sqrt{x+5}\sqrt{x-3} +(x-3) = 25$$
Förenkla vänsterledet
$$5x + 17 - 4\sqrt{x+5}\sqrt{x-3} +(x-3) = 25$$
Isolerar den kvarvarande rottermen  $ - 4\sqrt{x+5}\sqrt{x-3}$
$$- 4\sqrt{x+5}\sqrt{x-3}= 25 - 5x - 17 $$
Förenklar sedan högerledet
$$- 4\sqrt{x+5}\sqrt{x-3} = 8 - 5x$$
Kvadrera båda led för att få bort det kvarvarande rottecknet
$$ (-4\sqrt{x+5}\sqrt{x-3})^2 = (8 - 5x)^2$$
$$ 16(x^2 + 2x -15) = 25x^2-80x+64$$
$$ 16x^2 + 32x - 240 = 25x^2-80x+64$$
För över alla termer till högerledet och förenkla det
$$ 0 = 25x^2-80x+64 - 16x^2 - 32x + 240$$
$$ 0 = 9x^2-112x+302$$
\subsection{Kvadratkompletering}
Nu när vi har fått ut andragradsekvationen $$9x^2-112x+302=0$$ kan vi använda oss av rotformeln för andragrads ekvationer:
$$ax^2 + bx + c = 0 \Rightarrow x_{1,2} = \frac{-b \pm \sqrt{b^2 - 4ac}}{2a}$$
som kan härledas från pq-formeln som finns på sida 55-56 i Algebra 1$^1$.
Det betyder att
$$9x^2-112x+302 = 0 \Rightarrow x_{1,2} = \frac{-(-112) \pm \sqrt{(-112)^2 - 4 * 9 * 302}}{2 * 9}$$
Om vi förenklar detta får vi
$$\Rightarrow x_{1,2} = \frac{ 112 \pm 40}{18}$$
Och då kan vi lösa ut $x_1$ och $x_2$
$$x_1 = \frac{ 112 - 40}{18} = \frac{72}{18} = 4$$
$$x_2 = \frac{ 112 + 40}{18} = \frac{152}{18} = 8.444\dots =8 + \frac{4}{9}$$
\subsection{Validera rötter}
  Vi kan validera om rötterna stämmer igenom att substituera $x$ med $x_1$ och $x_2$ i ursprungs formeln och se om vi får en motsägelse.\\
$x = x_1 = 4$ ger  
  $$2\sqrt{4+5}-\sqrt{4-3} = 5$$
  $$2\sqrt{9}-\sqrt{1} = 5$$
  $$2*3-1 = 5$$
  $$6 - 1  = 5$$
  $$5 = 5$$
  $x_1$ är då inte en falsk rot\\
  $x = x_2 = 8 + \frac{4}{9}$ ger  
  $$2\sqrt{8 + \frac{4}{9}+5}-\sqrt{8 + \frac{4}{9}-3} = 5$$
  $$2\sqrt{13.444\dots}-\sqrt{5.444\dots} = 5$$
  $$2*3.666\dots - 2.333\dots = 5$$
  $$7.333\dots - 2.333\dots = 5$$
  $$4.999\dots = 5$$
  $$5 = 5$$
  $x_2$ är då inte en falsk rot\\
  
\subsection{Svar}
Lösningarna är på rotekvationen $2\sqrt{x+5}-\sqrt{x-3} = 5$ är:
$$x_1 = 4$$
$$x_2 = 8 + \frac{4}{9}$$
\section{Diskussion}

En observation man kan göra är att den ursprunliga ekvationen innehåller $\sqrt{x-3}$ vilket betyder att om $x < 3 $ kommer $\sqrt{x-3}$ att bli roten av något negativt, vilket blir ett imaginärt tal. Vi har inte heller någon annan term som kan bilda ett imaginärt tal så de imaginära delarna skulle eliminera varnadra. Detta betyder att $x \geq 3 $.\\
I detta fall hjälpte inte denna observation att lösa ut rötterna då både $x_1$ och $x_2$ är större än 3, men i ett liknande exempel skulle detta kunna eliminera en falsk rot.
\section{Referenser}
1 Bøgvard, Rickard och Xantcha, Qimh. \textit{Algebra 1}. Nionde tryckningen. Stockholms universitet. 2016.
\end{document}
